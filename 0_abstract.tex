\prefacesection{Abstract}

Solution of linear systems is often the bottleneck in large-scale subsurface modeling, especially when tightly coupled multiphysics systems are considered.   Iterative linear solvers that are employed critically depend on efficient and scalable preconditioners to achieve robust convergence.   Multilevel, and in particular multiscale, methods are among some of the best available techniques for reducing computational complexity and accelerating  the solution of linear systems, and have enjoyed a lot of success particularly in the domain of porous media flow problems.   The objective of this work is the development of scalable and efficient multilevel multiscale preconditioners targeting problems of geomechanics, single-phase flow and poromechanics on general polyhedral grids in one unified framework.

Based on the aforementioned requirements, we have developed a flexible unstructured coarsening framework that constructs a multilevel hierarchy of topological grid representations suitable for both nodal (e.g. finite-element) and cell-centered (e.g. finite-volume) discretizations.   At every level, interpolation and restriction operators are defined by computing numerical basis functions using the Multiscale Restriction-Smoothed Basis (MsRSB) method, which has been additionally extended with a matrix filtering strategy designed to guarantee convergence of the smoothing iterations for problems that do not produce an M-matrix.   Our framework defines a compact support for each basis function that avoids the growth of stencil size at coarser levels.   Combined with suitable smoothers, the multilevel multiscale preconditioner effectively tackles multiple modes of the error spectrum.   We also develop a multiscale preconditioner for coupled poromechanical problems based on a combination of pressure and displacement basis functions and a block-triangular smoothing operator.    Robustness and algorithmic scalability of the method are verified using a series of two and three-dimensional test cases including single-physics and coupled problems.   Additionally, extensions of the framework to multiphase poromechanical problems and handling of wells are proposed.

This work also describes the design and several important components of GEOSX, a high-performance multiphysics simulation framework.   We detail some of our contributions to the framework, including the development of an accelerated multi-component multiphase flow and transport solver, for which several programming techniques are described that are necessary to achieve good performance of physics kernels on GPU architectures.   Further, we describe the parallel programming model of GEOSX as well as components of the flexible linear algebra interface layer that form the basis for the development of a multiscale solver within the simulation framework.

Finally, we describe the optimized parallel implementation of the proposed multiscale methods.   We describe the main computational kernels and our approach to parallel execution for some of them, as well as choices of data structures for representing the mesh hierarchy and multiscale operators.   We investigate the performance and scalability of the developed multiscale solver on shared (multi-core) and distributed memory architectures and compare it to a state-of-the-art classical algebraic multigrid solver using a series of structured and unstructured grid test problems.   Good weak and strong scaling is observed on up to 16 cores and 32 cluster nodes, and for geomechanical problems the multiscale solver is found to outperform AMG.