\chapter{Introduction}
\label{ch:introduction}

% ===================== SECTION =====================
\section{Motivation}

Understanding, quantifying and predicting the behavior of subsurface poromechanical systems is of critical importance in many applications in the energy industry and beyond, from geological storage of CO\textsubscript{2}, to oil and gas production, to geothermal reservoirs, to management of groundwater resources.   In many of these applications, fast and accurate predictions are necessary to optimize the cost of operation, assess uncertainties and manage risks associated with complex interactions between various physical phenomena, including land subsidence, induced seismicity, fault reactivation, seal and wellbore integrity.   With the increasing cost and complexity of these projects, simple analytical models no longer provide the necessary level of detail, and numerical modeling has become an indispensable tool aiding researchers, engineers and regulatory agencies alike.

Subsurface numerical models are constantly growing in their size and complexity, placing an ever increasing demand for robust and scalable computational methods and tools.   Typical reservoir models are characterized by large spatial (tens of kilometers) and temporal (tens to hundreds of years) extents as well as heterogeneous and anisotropic petrophysical properties (such as porosity, permeability and mechanical moduli) possessing multi-scale correlation structures that require high-fidelity geological models in order to resolve the important fine-scale details.   Consequently, a simulation tool must be able to perform thousands of time steps with tens of millions of unknowns to solve for at every step in a reasonable amount of time.   Many of these highly-detailed models have abandoned the computationally attractive framework of logically structured Cartesian grids and are described using partially or fully unstructured meshes, owing to their ability to accurately represent important geological and stratigraphic features such as faults and pinch-outs.   Some of the solution approaches previously developed may not be applicable to these complex meshes, requiring more flexible techniques to be sought.

In addition, the complex nature of physical phenomena involved requires solving multiple governing equations that are often tightly coupled, leading to systems with degrees-of-freedom of mixed nature, that often require specialized algorithms and solution schemes to deal with.   In recent years, simulations involving coupled geomechanics, porous media flow, heat transfer, multiphase multi-component transport and chemical reactions have begun to emerge.   Moreover, with a large amount of uncertainty that is inherently present in subsurface models due to limited observations available, statistically generated ensembles consisting of thousands of realizations and dozens of modeling scenarios must be simulated to allow for data-driven decision making.   All of these requirements present a formidable challenge to existing and upcoming modeling tools. 

A large number of techniques have been developed to reduce the size of models and make them more feasible for both forward and inverse modeling, including upscaling as well surrogate and reduced-order models.   Most recently, advancements in supervised machine learning made it possible to train large-scale neural network models to predict behavior of subsurface systems by generalizing observations and extracting patterns from simulation results.   We note that whether or not these models are ultimately successful as prediction tools, they still need large volumes of training data that must be generated by physics-driven simulation.   As such, the demand for fast and accurate forward modeling software continues to increase.

At the same time, improvements in design and manufacturing of processing units have approached the physical limits of a single chip, and all modern computing hardware is increasingly parallel at every level, from multi-core CPUs to GPU (graphical processing units) accelerators featuring thousands of cores, to wafer-scale systems, to computing clusters with hundreds of nodes.   In order to keep up with the computational requirements, existing techniques must be continuously improved and new scalable algorithms must be developed that are capable of taking advantage of parallel hardware efficiently.   In the area of subsurface modeling, multi-level and multi-scale methods have been shown to be particularly attractive as robust and high-performing solvers and preconditioners.

% ===================== SECTION =====================
\section{Problem Statement}

\subsection{Governing Equations and Discrete Problem}

\subsection{Solution Approaches for Two-way Coupled Systems}

\subsection{Iterative Solvers and Preconditioning}

% ===================== SECTION =====================
\section{Dissertation Outline}